\begin{SCn}

\input{Contents/toc}

\newpage

\scseparatedfragment{Титульная спецификация Стандарта OSTIS}

\begin{SCn}

\scnsectionheader{\currentname}

\scnstartsubstruct

\scnrelto{титульная спецификация}{Стандарт OSTIS}
\scnaddlevel{1}
	\scnrelfrom{оглавление}{Оглавление Стандарта OSTIS}
	\scnrelfrom{общая структура}{Общая Структура Стандарта OSTIS}
	\scnrelfrom{система ключевых знаков}{Система ключевых знаков Стандарта OSTIS}
	\scnrelfrom{редакционная коллегия}{Редакционная коллегия Стандарта OSTIS}
	\scnrelfrom{авторский коллектив}{Авторский коллектив Стандарта OSTIS}
	\scnrelfrom{направления развития}{Направления развития Стандарта OSTIS}
	\scnrelfrom{правила построения}{Правила построения Стандарта OSTIS}
	\scnaddlevel{1}
		\scnidtf{правила построения*(Стандарт OSTIS)}
		\scnaddlevel{1}
			\scniselement{sc-выражение}
		\scnaddlevel{-1}
	\scnaddlevel{-1}
	\scnrelfrom{правила организации развития}{Правила организации развития Стандарта OSTIS}
	\scnaddlevel{1}	
		\scnrelfromset{декомпозиция}{Правила организации развития исходного текста Стандарта OSTIS;Правила организации развития Стандарта OSTIS на уровне его внутреннего представления в памяти Метасистемы IMS.ostis}
\scnaddlevel{-2}
\scnauthorcomment{Вычитать то, что дальше}		
\scneqfile{В титульную спецификацию \textit{Стандарта OSTIS} должны быть включены ссылки на все разделы и фрагменты этих разделов, где описываются правила построения и оформления всех видов информационных конструкций, входящих в состав \textit{Стандарта OSTIS} (внешних идентификаторов знаков, входящих в состав \textit{Стандарта OSTIS}, спецификаций различного вида сущностей,описываемых в \textit{Стандарте OSTIS})\\
В \textit{базах знаний ostis-систем} задаются правила унифицированного построения (представления оформления) следующих видов \textit{информационных конструкций}:
\begin{scnitemize}
	\item\textit{sc-идентификаторов} - внешних идентификаторов \textit{sc-элементов} следующих классов:
	\begin{itemize}
		\item\textit{sc-элементов}(имеются в виду общие правила идентификации любых sc-элементов)- смотрите в разделе ""
		\item\textit{sc-переменных,sc-констант}
		\item знаков майф. сущности
		\begin{itemize}
			\item знаков персон
			\item знаков библ.-источников 
		\end{itemize}
		\item знаков множеств 
		\begin{itemize}
			\item классов,понятий
			\begin{itemize}
				\item отношений
				\item параметров
				\item стр-р
				\begin{itemize}
					\item знаний
				\end{itemize}
			\end{itemize}
		\end{itemize}
		\item знаков файлов ostis-систем
		\item знаков sc-знаний баз знаний 
	\end{itemize}
	\item\textit{sc-конструкций}
	\item\textit{sc.g-конструкций}
	\item\textit{sc.s-конструкций}
	\item\textit{sc.n-конструкций}
	\item базовых правил \textit{sc-спецификаций:}
	\begin{itemize}
		\item понятий 
		\item разделов баз знаний (титульные спецификации разделов)
		\item файлов ostis-систем
		\item библ. источников
		\item предметных областей 	
	\end{itemize}
	\item специализированная \textit{sc-спецификация}
	\begin{itemize}
		\item информационная конструкция
		\begin{itemize}
			\item оглавление
			\item система ключевых знаков	
		\end{itemize}
	\end{itemize}
	\begin{itemize}
		\item понятий
		\begin{itemize}
			\item пояснение
			\item определения
			\item теоретико-множественная окрестность
			\item сем-во утверждений	
		\end{itemize}
	\end{itemize}
	\begin{itemize}
		\item сегментов баз знаний(титульная спецификация)
		\item семейство разделов баз знаний	
	\end{itemize}
\end{scnitemize}}
\scnaddlevel{2}

\scnheader{Стандарт OSTIS-2021}
\scnidtf{Издание Документации Технологии OSTIS-2021}
\scnidtf{Первое издание (публикация) Внешнего представления Документации Технологии OSTIS в виде книги}
\scniselement{публикация}
\scnaddlevel{1}
\scnidtf{библиографический источник}
\scnaddlevel{-1}
\scniselement{официальная версия Стандарта OSTIS}
\scniselement{бумажное издание}
\scniselement{научное издание}
\scnrelfrom{рекомендация издания}{Совет БГУИР}
\scnrelfromset{рецензенты}{Курбацкий А.Н.; Дудкин А.А.}
\scnrelfrom{издательство}{Бестпринт}
\scniselement{\scnstartsetlocal\scnendstructlocal}
\scnaddlevel{1}
\scniselement{УДК}
\scnaddlevel{1}
\scniselement{параметр}
\scnaddlevel{-1}
\scnrelfrom{Индекс УДК}{004.8}
\scnaddlevel{-1}
\scnidtftext{ISBN}{978-985-7267-13-2}

\scnheader{Стандарт OSTIS-2022}
\scnidtf{Издание Документации Технологии OSTIS-2022}
\scnidtf{Второе издание (публикация) Внешнего представления Документации Технологии OSTIS в виде книги}
\scniselement{публикация}
\scniselement{официальная версия Стандарта OSTIS}
\scniselement{бумажное издание}
\scniselement{научное издание}

\bigskip
\scnendstruct \scninlinesourcecommentpar{Завершили Титульную спецификацию \textit{Стандарта OSTIS}}

\end{SCn}

\newpage

\input{Contents/title_part_2022}

\newpage

\scnsectionheader{Правила оформления}

\begin{SCn}
	
\scnheader{Стандарт OSTIS}
\scnrelfrom{общие правила построения}{\scnkeyword{Общие правила построения Стандарта OSTIS}}
	\scnaddlevel{1}
	\scnidtf{принципы лежащие в основе структуризации и оформления Стандарта OSTIS}
	\scneq{
	\scnmakevectorlocal{\scnfileitem{Основной формой представления \textit{Стандарта OSTIS} как полной документации текущего состояния \textit{Технологии OSTIS} является \textit{текущее состояние} основной части \textit{базы знаний} специальной интеллектуальной компьютерной \textit{Метасистемы IMS.ostis}, обеспечивающей использование и эволюцию (перманентное совершенствование) \textit{Технологии OSTIS}. Такое представление \textit{Стандарта OSTIS} обеспечивает эффективную семантическую навигацию по содержанию \textit{Стандарта OSTIS} и возможность задавать \textit{Метасистеме IMS.ostis} широкий спектр нетривиальных вопросов о самых различных деталях и тонкостях \textit{Технологии OSTIS}};
	\scnfileitem{Непосредственно сам \textit{Стандарт OSTIS} представляет собой внутреннее \textit{смысловое представление} основной части базы знаний \textit{Метасистемы IMS.ostis} на внутреннем смысловом языке \textit{ostis-систем} (этот язык назван нами \textit{SC-кодом} - Semantic Computer Code)};   
	\scnfileitem{С семантической точки зрения \textit{Стандарт OSTIS} представляет собой иерархическую систему формальных моделей \textit{предметных областей} и соответствующих им \textit{формальных онтологий}};
	\scnfileitem{С семантической точки зрения \textit{Стандарт OSTIS} представляет собой большую \textit{рафинированную семантическую сеть}, которая, соответственно, имеет нелинейный характер и которая включает в себя знаки любых видов описываемых сущностей(материальных сущностей, абстрактных сущностей, понятий, связей, структур) и, соответственно этому, содержит связи между всеми этими видами сущностей(в частности, связи между связями, связи между структурами)};
	\scnfileitem{В состав Стандарта OSTIS входят также файлы информационных конструкций, не являющихся конструкциями SC-кода (в том числе и sc-текстов,принадлежащих различным естественным языкам). Такие файлы позволяют формально описывать в базе знаний синтаксис и семантику различных внешних языков, а также позволяют включать в состав базы знаний различного рода пояснения, примечания, адресуемые непосредственно пользователям и помогающие им в понимании формального текста базы знаний};
	\scnfileitem{Кроме представления \textit{Стандарта OSTIS} на внутреннем \textit{языке представления знаний} используется также внешняя форма представления \textit{Стандарта OSTIS} на \textit{внешнем языке представления знаний} (на языке исходных текстов знаний и баз знаний). При этом указанное внешнее представление \textit{Стандарта OSTIS} должно быть структурировано и оформлено так,чтобы читатель мог достаточно легко "вручную{}" найти в этом тексте практически любую интересующую его \textit{информацию}. В качестве \textit{формального языка} внешнего представления \textit{Стандарта OSTIS} используется \textit{SCn-код}, описание которого приведено в \textit{Стандарте OSTIS} в разделе "\nameref{intro_scn}{}"};
	\scnfileitem{Предлагаемое Вам издание \textit{Стандарта OSTIS} представляется на формальном языке \textit{SCn-код}, который является языком внешнего представления больших текстов \textit{SC-кода}, в которых большое значение имеет наглядная структуризация таких текстов.};
	\scnfileitem{\textit{Стандарт OSTIS} имеет онтологическую структуризацию, т.е. представляет собой иерархическую систему связанных между собой \textit{формальных предметных областей} и соответствующих им \textit{формальных онтологий}.		
	Благодаря этому обеспечивается высокий уровень стратифицированности \textit{Стандарта OSTIS}};
	\scnfileitem{Каждому \textit{понятию}, используемому в \textit{Стандарте OSTIS}, соответствует свое место в рамках этого Стандарта,своя \textit{предметная область} и соответствующая ей \textit{онтология}, где это \textit{понятие} подробно рассматривается (исследуется), где концентрируется вся основная информация об этом \textit{понятии}, различные его свойства};
	\scnfileitem{Кроме \textit{Общих правил построения Стандарта OSTIS}. В \textit{Стандарте OSTIS} приводятся описания различных гостных (специализированных) правил построения (оформления) различных видов фрагментов \textit{Стандарта OSTIS}.	
	К таким видам фрагментов относятся следующие:
		\begin{scnitemize}
			\item\textit{sc-идентификатор}
			\scnaddlevel{-3}	
				\scnidtf{внешний идентификатор внутреннего знака (\textit{sc-элемента}) входящего в состав \textit{базы знаний ostis-системы}}	
				\scnidtf{\textit{информационная конструкция} (чаще всего это строка символов) обеспечивающая однозначную идентификацию соответствующей сущности,описываемой в \textit{базах знаний ostis-систем}, и являющаяся, чаще всего, именем(термином), соответствующим описываемой сущности,именем, обозначающим эту сущность во внешних текстах \textit{ostis-систем}}
			\scnaddlevel{3}
			\item\textit{sc-спецификация}
			\scnaddlevel{-3}
				\scnidtf{семантическая окрестность}
				\scnidtf{семантическая окрестность соответствующего \textit{sc-элемента} (внутреннего знака, хранимого в памяти \textit{ostis-системы} в составе её \textit{базы знаний},представленной на внутреннем языке \textit{ostis-систем.})}
				\scnidtf{семантическая окрестность некоторого \textit{x-элемента}, хранимого в \textit{sc-памяти}, в рамках текущего состояния этой \textit{памяти}}
			\scnaddlevel{3}
			\item(\textit{sc-конструкция} $\setminus$ \textit{sc-спецификация})
			\scnaddlevel{-3}
				\scnidtf{\textit{sc-конструкция} (конструкция \textit{SC-кода} - внутреннего языка \textit{ostis-систем}), не являющаяся \textit{sc-спецификацией}}
				\scnaddlevel{1}
				\scnrelfrom{сокращение}{\scnfilelong{\textit{sc-конструкция}, не являющаяся \textit{sc-спецификацией}}}
				\scnaddlevel{-1}
			\scnaddlevel{3}
			\item(\textit{файл ostis-системы $\setminus$ sc-идентификатор})
			\scnaddlevel{-3}
				\scnidtf{\textit{файл ostis-системы}, не являющийся sc- идентификатором}
			\scnaddlevel{3}
		\end{scnitemize}};
	\scnfileitem{Правила построения \textit{sc-идентификаторов} будем также называть Правилами внешней идентификации sc-элементов.
	\textit{Общие правила построения sc-идентификаторов} смотрите в разделе ''\nameref{intro_idtf}''.
	В состав этих правил входят правила внешней идентификации \textit{sc-констант, sc-переменных},\textit{отношений, параметров},\textit{sc-конструкций, файлов ostis-систем.}};
	\scnfileitem{К числу частных правил построения sc-идентификаторов относятся:
		\begin{scnitemize}
			\item\textit{Правила построения sc-идентификаторов персон} -- смотрите раздел ''\nameref{sd_person}''.
			\item\textit{Правила построения sc-идентификаторов библиографических источников} - смотрите раздел ''\nameref{sd_bibliography}''.
		\end{scnitemize}};
	\scnfileitem{Общие правила построения \textit{sc-конструкций} (конструкций \textit{SC-кода} - внутреннего языка ostis-систем) смотрите в разделе ''\nameref{intro_sc_code}'', а также в разделе ''\nameref{sd_sc_code_syntax}'' и в разделе ''\nameref{sd_sc_code_semantic}''};
	\scnfileitem{К числу частных правил построения \textit{sc-конструкций} относятся:
		\begin{scnitemize}
			\item\textit{Правила построения баз знаний ostis-систем} -- смотрите раздел ''\nameref{sd_knowledge}''
			Эти правила направлены на обеспечение целостности баз знаний ostis-систем, на обеспечение (1) востребованности (нужности) знаний, входящих в состав каждой базы знаний, и (2) целостности самой базы знаний, т.е. достаточности знаний, входящих в состав каждой базы знаний для эффективного функционирования соответствующей ostis-системы
			\item\textit{Правила построения разделов и сегментов баз знаний ostis-систем} смотрите раздел
			\item\textit{Правила представления логических формул и высказываний в базах знаний ostis-систем}
			\item\textit{Правила представления формальных предметных областей в базах знаний ostis-систем}
			\item\textit{Правила представления формальных логических онтологий в базах знаний ostis-систем}
		\end{scnitemize}};
	\scnfileitem{Формальное описание синтаксиса и семантики \textit{SCn-кода} приведено в разделе ''\nameref{sd_scp_syntax }''}}}
\scnauthorcomment{Вычитать то, что дальше}
\textit{сослаться на:}
\begin{scnitemize}
	\item правила построения sc-идентификаторов для различных классов sc-элементов
	\item правила построения различного вида sc-текстов
	\begin{itemize}
		\item разделов базы знаний
		\item баз знаний	
	\end{itemize}	
\end{scnitemize}
\begin{scnitemize}
	\item правила построения sc-спецификаций сущностей
	\begin{itemize}
		\item разделов базы знаний
		\item предметная область
		\item сегментов базы знаний
		\item библ.- источников	
	\end{itemize}	
\end{scnitemize}
\begin{scnitemize}
	\item типичные опции
	\begin{itemize}
		\item баз знаний	
	\end{itemize}	
\end{scnitemize}
\begin{scnitemize}
	\item общие направления развития
	\begin{itemize}
		\item раздел базы знаний	
	\end{itemize}	
\end{scnitemize}
\begin{scnitemize}
	\item направления развития*:
	\begin{itemize}
		\item Стандарта OSTIS
		\item Введение в язык OSTIS-Cn	
		\item Предметная область и онтология библиографии
		\item Библиография OSTIS
	\end{itemize}	
\end{scnitemize}
\begin{scnitemize}
	\item направления и правила	
	\begin{itemize}
		\item деятельность	
		\item Редк. Стандарта OSTIS
		\item соавтор Стандарта OSTIS
	\end{itemize}
\end{scnitemize}

	
\end{SCn}

\newpage



\bigskip
\scnfragmentcaption

\scnheader{Пояснения к оглавлению Стандарта OSTIS и к некоторым разделам этого Стандарта}

\scnstartsubstruct

\scnheader{Спецификация второго издания Стандарта OSTIS}
\scnidtf{Спецификация второй официальной версии Стандарта OSTIS}
\scnidtf{Спецификация Стандарта OSTIS-2022}

\scnheader{Анализ методологических проблем современного состояния работ в области Искусственного интеллекта}
\scnidtf{Актуальность Технологии OSTIS}
\scnidtf{Современные требования, предъявляемые к деятельности в области Искусственного интеллекта  к интеллектуальным компьютерным системам следующего поколения -- конвергенция, глубокая ("бесшовная"{}) интеграция, высокий уровень обучаемости (гибкости, стратифицированности, рефлексивности), высокий уровень социализации (взаимопонимания, договороспособности, способности координировать свои действия с другими субъектами), стандартизация}

\scnheader{Введение в описание внутреннего языка ostis-систем}
\scnidtf{Введение в SC-code (Semantic Computer Code)}

\scnheader{Предметная область и онтология внешних идентификаторов знаков, входящих в информационные конструкции внутреннего языка ostis-систем}
\scnidtf{Предметная область и онтология sc-идентификаторов}

\scnheader{Введение в описание языка графического представления информационных конструкций, хранимых в памяти ostis-систем}
\scnidtf{Введение в SCg-code (Semantic Code graphical)}

\scnheader{Введение в описание языка линейного представления информационных конструкций, хранимых в памяти ostis-систем}
\scnidtf{Введение в SCs-code (Semantic Code string)}

\scnheader{Введение в описание языка форматирования линейного представления информационных конструкций, хранимых в памяти ostis-систем}
\scnidtf{Введение в SCn-code (Semantic Code natural)}

\scnheader{Предметная область и онтология кибернетических систем}
\scnidtf{Предпосылки создания компьютерных систем нового поколения}

\scnheader{Предметная область и онтология компьютерных систем}
\scnidtf{Этапы эволюции (повышения качества) компьютерных систем -- эволюции памяти, информации, хранимой в памяти, решателей задач, интерфейсов}

\scnheader{Предметная область и онтология интеллектуальных компьютерных систем}
\scnidtf{Этапы эволюции (повышения качества) интеллектуальных компьютерных систем и проблемы дальнейшей их эволюции}

\scnheader{Предметная область и онтология технологий автоматизации различных видов человеческой деятельности}
\scnidtf{Эволюция технологий проектирования, производства и эксплуатации компьютерных систем и предпосылки создания компьютерных технологий нового поколения}

\scnheader{Предметная область и онтология логико-семантических моделей компьютерных систем, основанных на смысловом представлении информации}
\scnidtf{Предлагаемый подход к построению интеллектуальных компьютерных систем следующего поколения}

\scnheader{Предметная область и онтология внутреннего языка ostis-систем}
\scnidtf{Предметная область и онтология SC-кода (Semantic Computer Code)}
\scnrelfrom{введение}{\textit{\nameref{intro_sc_code}}}

\scnheader{Предметная область и онтология  базовой денотационной семантики SC-кода}
\scniselement{\textit{предметная область и онтология верхнего уровня}}


\scnheader{Предметная область и онтология языка графического представления информационных конструкций, хранимых в памяти ostis-систем}
\scnidtf{Предметная область и онтология SCg-кода (Semantic Code graphical)}
\scnaddhind{1}
\scnrelfrom{введение}{\textit{\nameref{intro_scg}}}
\scnresetlevel

\scnheader{Предметная область и онтология языка линейного представления информационных конструкций, хранимых в памяти ostis-систем}
\scnidtf{Предметная область и онтология SCs-кода (Semantic Code string)}
\scnaddhind{1}
\scnrelfrom{введение}{\textit{\nameref{intro_scs}}}
\scnresetlevel

\scnheader{Предметная область и онтология языка форматирования линейного представления информационных конструкций, хранимых в памяти ostis-систем}
\scnidtf{Предметная область и онтология SCn-кода (Semantic Code natural)}
\scnaddhind{1}
\scnrelfrom{введение}{\textit{\nameref{intro_scn}}}
\scnresetlevel

\scnheader{Предметная область и онтология файлов, внешних информационных конструкций и внешних языков ostis-систем}
\scnrelto{дочерний раздел}{\nameref{intro_lang}}
\scnheader{sc-идентификатор файла ostis-системы}
\scnrelfrom{правила построения}{Правила идентификации файлов ostis-систем}
\scnheader{файл ostis-системы}
\scnrelfrom{sc-файл ostis-системы:}{}
\scnheader{sc-файл ostis-системы}
\scnrelfrom{правила построения:}{Правила построения sc-файлов ostis-систем}
\scnheader{спецификация файла ostis-системы}
\scnrelfrom{правила построения:}{Правила спецификации файлов ostis-систем}


\scnheader{Предметная область и онтология операционной семантики sc-языка вопросов}
\scnidtf{Предметная область информационно-поисковых действий и агентов, а также соответствующая онтология методов}

\scnheader{Предметная область и онтология операционной семантики логических sc-языков}
\scnidtf{Предметная область и онтология логических исчислений}
\scnidtf{Предметная область и онтология действий и агентов логического вывода, а также соответствующая онтология методов (правил) логического вывода}

\scnheader{Предметная область и онтология sc-языков программирования высокого уровня}
\scnidtf{Предметная область и онтология sc-языков программирования высокого и сверхвысокого уровня, ориентированных на обработку баз знаний ostis-систем}

\scnheader{Предметная область и онтология операционной семантики sc-моделей искусственных нейронных сетей}
\scnidtf{Предметная область и онтология процессов функционирования sc-моделей искусственных нейронных сетей при обработке баз знаний ostis-систем}

\scnheader{Логико-семантическая модель средств автоматизации управления взаимодействием разработчиков различных категорий в процессе проектирования базы знаний ostis-системы}
\scnidtf{Логико-семантическая модель средств автоматизации управления взаимодействием менеджеров, авторов, рецензентов, экспертов и редакторов в процессе проектирования базы знаний ostis-системы}

\scnheader{Предметная область и онтология встроенных ostis-систем поддержки эксплуатации соответствующих ostis-систем конечными пользователями}
\scnidtf{Интеллектуальные \textit{встроенные ostis-системы}, обучающие \textit{конечных пользователей} эффективной эксплуатаии тех \textit{ostis-систем}, в состав которых они входят}
\scnidtf{Предметная область и онтология методов и средств реализации целенаправленного и персонифицированного обучения пользователей каждой ostis-системы}

\scnheader{Предметная область и онтология Экосистемы OSTIS}
\scnidtf{Проект smart-общества}

\scnheader{Логико-семантическая модель Метасистемы IMS.ostis}
\scnrelfrom{примечание}{\scnstartsetlocal

	\bigskip
	\scnfilelong{IMS.ostis}
	\scnrelto{сокращение}{\scnfilelong{Метасистема IMS.ostis}}
	\scnaddlevel{1}
	\scnrelto{сокращение}{\scnfilelong{Intelligent MetaSystem of Open Semantic Technology for Intelligent Systems}}
	\scnaddlevel{-1}
	\scnendstruct}
\scnidtf{Логико-семантическая модель интеллектуального ostis-портала научно-технических знаний по Технологии OSTIS}

\scnendstruct

\end{SCn}

\newpage